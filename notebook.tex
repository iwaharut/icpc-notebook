\documentclass[a4paper, landscape, 9pt]{jarticle} % フォントサイズを9ptに設定
\usepackage{listings}
\usepackage{jlisting}
\usepackage{xcolor}
\usepackage[top=0.5in, bottom=0.5in, left=0.5in, right=0.5in]{geometry} % マージンを設定
\usepackage{multicol} % 複数列のためのパッケージ

% C++スタイルの定義
\lstdefinestyle{cpp}{
    language=C++,
    basicstyle=\ttfamily\footnotesize,
    keywordstyle=\bfseries\color{green!40!black},
    commentstyle=\itshape\color{red!80!black},
    stringstyle=\color{purple!40!black},
    identifierstyle=\color{blue},
    numbers=none, % 行番号を表示しない
    backgroundcolor=\color{gray!10},
    frame=single,
    breaklines=true,
    captionpos=b,
    tabsize=4,
    showspaces=false,
    showstringspaces=false
}

% Pythonスタイルの定義
\lstdefinestyle{python}{
    language=Python,
    basicstyle=\ttfamily\footnotesize,
    keywordstyle=\bfseries\color{blue},
    commentstyle=\itshape\color{green!40!black},
    stringstyle=\color{red},
    identifierstyle=\color{black},
    numbers=none, % 行番号を表示しない
    backgroundcolor=\color{gray!10},
    frame=single,
    breaklines=true,
    captionpos=b,
    tabsize=4,
    showspaces=false,
    showstringspaces=false,
    morekeywords={self} % Pythonのキーワードに'self'を追加
}

\lstset{style=cpp} % デフォルトのスタイルをC++に設定

\title{\vspace{-4ex}\huge{ICPC template}} % タイトルを設定
\author{} % 著者名を設定
\date{} % 日付を設定

\begin{document}



\begin{multicols*}{3} % 3列に分割(*を使うと全ページで適用)
    \maketitle
    \vspace{-25mm}
    \section*{Example C++ Code 1}
    \begin{lstlisting}
    #include <iostream>
    using namespace std;

    int main() {
        cout << "Hello, World!" << endl;
        return 0;
    }
    \end{lstlisting}

    \section*{Example C++ Code 2}
    \begin{lstlisting}
    #include <vector>
    #include <algorithm>
    using namespace std;

    int main() {
        vector<int> v = {1, 2, 3, 4, 5};
        sort(v.begin(), v.end(), greater<int>());
        for (int i : v) {
            cout << i << " ";
        }
        cout << endl;
        return 0;
    }
    \end{lstlisting}

    \section*{Example C++ Code 3}
    \begin{lstlisting}
    #include <map>
    #include <string>
    using namespace std;

    int main() {
        map<string, int> ages;
        ages["Alice"] = 30;
        ages["Bob"] = 25;
        ages["Charlie"] = 35;

        for (const auto &entry : ages) {
            cout << entry.first << ": " << entry.second << endl;
        }
        return 0;
    }
    \end{lstlisting}


    % Pythonのセクションに切り替える
    \lstset{style=python}

    \section*{Example Python Code 1}
    \begin{lstlisting}
    def hello_world():
        print("Hello, World!")

    if __name__ == "__main__":
        hello_world()
    \end{lstlisting}

    \section*{Example Python Code 2}
    \begin{lstlisting}
    def sort_descending(numbers):
        return sorted(numbers, reverse=True)

    numbers = [1, 2, 3, 4, 5]
    sorted_numbers = sort_descending(numbers)
    print(sorted_numbers)
    \end{lstlisting}

    \section*{Example Python Code 3}
    \begin{lstlisting}
    class Person:
        def __init__(self, name, age):
            self.name = name
            self.age = age

        def greet(self):
            print(f"Hello, my name is {self.name} and I am {self.age} years old.")

    alice = Person("Alice", 30)
    alice.greet()
    \end{lstlisting}

\end{multicols*}

\end{document}
