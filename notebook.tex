\documentclass[a4paper, landscape, 9pt]{jarticle} % フォントサイズを9ptに設定
\usepackage{listings}
\usepackage{jlisting}
\usepackage{xcolor}
\usepackage[top=0.5in, bottom=0.5in, left=0.5in, right=0.5in]{geometry} % マージンを設定
\usepackage{multicol} % 複数列のためのパッケージ


% C++スタイルの定義
\lstdefinestyle{cpp}{
    language=C++,
    basicstyle=\ttfamily\footnotesize,
    keywordstyle=\bfseries\color{green!40!black},
    commentstyle=\itshape\color{red!80!black},
    stringstyle=\color{purple!40!black},
    identifierstyle=\color{blue},
    numbers=none, % 行番号を表示しない
    backgroundcolor=\color{gray!10},
    frame=single,
    breaklines=true,
    captionpos=b,
    tabsize=4,
    showspaces=false,
    showstringspaces=false
}

% Pythonスタイルの定義
\lstdefinestyle{python}{
    language=Python,
    basicstyle=\ttfamily\footnotesize,
    keywordstyle=\bfseries\color{blue},
    commentstyle=\itshape\color{green!40!black},
    stringstyle=\color{red},
    identifierstyle=\color{black},
    numbers=none, % 行番号を表示しない
    backgroundcolor=\color{gray!10},
    frame=single,
    breaklines=true,
    captionpos=b,
    tabsize=4,
    showspaces=false,
    showstringspaces=false,
    morekeywords={self} % Pythonのキーワードに'self'を追加
}

\lstset{style=cpp} % デフ}ォルトのスタイルをC++に設定

\title{\vspace{-4ex}\huge{ICPC template}} % タイトルを設定
\author{} % 著者名を設定
\date{} % 日付を設定

\begin{document}



\begin{multicols*}{3} % 3列に分割(*を使うと全ページで適用)
    \maketitle
    \vspace{-25mm}
    \tableofcontents

    \newpage

    \part{C++}

    \section{C++ Preparation}
    \subsection{C++ Compiler}
    \begin{lstlisting}
    g++ -std=c++17 test.cpp
    \end{lstlisting}

    \subsection{C++ Execution}
    \begin{lstlisting}
    ./test.out <input> output
    \end{lstlisting}

    \subsection{C++ Template}
    \begin{lstlisting}
    #include <bits/stdc++.h>
    #include <atcoder/all>
    using namespace std;
    using namespace atcoder;
    using ll = long long;
    using ull = unsigned long long;
    using Graph = vector<vector<int>>;
    // int 2*10e9
    // long long 9*10e18
    // unsigned long long 1*10e19
    constexpr int INF = 1e9;
    constexpr ll LLINF = 4e18;
    #define for_(i,a,b) for(int i=(a);i<(b);++i)
    #define rep(i, n) for_(i, 0, n)
    #define all(a) (a).begin(), (a).end()
    #define rall(a) (a).rbegin(), (a).rend()

    //4方向
    int dx[4] = {1, 0, -1, 0};
    int dy[4] = {0, -1, 0, 1};

    //8方向
    int ddx[8] = {1,1,1,0,0,-1,-1,-1}; 
    int ddy[8] = {1,0,-1,1,-1,1,0,-1};

    int main() {
        
        return 0;
    }
    \end{lstlisting}

    \section{データ構造}
    \subsection{stack}
    \begin{lstlisting}
    stack<int> st;
    // [1, 2, 3]を追加
    st.push(1);
    st.push(2);
    cout << st.top() << endl; // 2
    st.push(3);
    st.pop(); // 3を削除
    st.pop(); // 2を削除
    cout << st.top() << endl; // 1
    if(st.empty()) // 空ならtrue
    \end{lstlisting}

    \subsection{queue}
    \begin{lstlisting}
    queue<int> que;
    // [1, 2, 3]を追加
    que.push(1);
    que.push(2);
    cout << que.front() << endl; // 1
    que.push(3);
    que.pop(); // 1を削除
    que.pop(); // 2を削除
    cout << que.front() << endl; // 3
    \end{lstlisting}

    \subsection{priority queue}
    \begin{lstlisting}
    // 最大値が先頭にくるキュー
    priority_queue<int> pq;
    // 最小値が先頭にくるキュー
    priority_queue<int,vector<int>,greater<int>> pq;
    // [1, 3, 5, 6]を追加
    pq.push(1);
    pq.push(3);
    cout << pq.top() << endl; // 3
    pq.push(5);
    pq.push(6);
    pq.pop(); // 6を削除
    cout << pq.top() << endl; // 5
    \end{lstlisting}

    \subsection{map}
    \begin{lstlisting}
    map<string, int> mp;
    mp["haru"] = 12;
    mp["taro"] = 13;
    cout << mp["haru"] << endl; // 12
    \end{lstlisting}

    \subsection{set}
    \begin{lstlisting}
    set<int> st;
    // [1, 2]を追加
    st.insert(1);
    st.insert(2);
    st.count(1); // 1が含まれていたら1を返す
    st.erase(1); // 1を削除
    cout << *st.begin() << endl; // 2
    \end{lstlisting}

    \subsection{tuple}
    \begin{lstlisting}
    tuple<int, string, long long> tp;
    tp = {20, "kindai", 100000};
    // 要素のアクセス
    cout << get<0>(tp) << endl; // 20
    cout << get<1>(tp) << endl; // kindai
    \end{lstlisting}

    \subsection{string}
    \begin{lstlisting}
    string S = "TUNA";
    // 文字列の長さを取得
    cout << S.size() << endl; // 4
    // 文字列が空か判定
    if(S.empty()) // 空ならtrue
    // 文字列の分割 1文字目以降を取得
    cout << S.substr(1) << endl; // UNA
    cout << S.substr(1, 2) << endl; // UN
    // 文字列の削除 1文字明光を削除
    cout << S.erase(1) << endl; // T

    \end{lstlisting}

    \subsection{Union-Find}
    \begin{lstlisting}
    // Union-Find
    // グリッドでUFを使う時,(x,y)に対して使うなら(x-1)*W+(y-1)でハッシュ化できる.
    struct UnionFind {
        vector<int> par, rank, siz;
        // 構造体の初期化
        UnionFind(int n) : par(n,-1), rank(n,0), siz(n,1) { }
        // 根を求める
        int root(int x) {
        if (par[x]==-1) return x;
        else return par[x] = root(par[x]);
        }
        // x と y が同じグループに属するか (= 根が一致するか)
        bool issame(int x, int y) {
            return root(x)==root(y);
        }
        // x を含むグループと y を含むグループを併合する
        bool unite(int x, int y) {
            int rx = root(x), ry = root(y);
            if (rx==ry) return false;
            // union by rank
            if (rank[rx]<rank[ry]) swap(rx, ry);
            par[ry] = rx; // ry を rx の子とする
            if (rank[rx]==rank[ry]) rank[rx]++;
            siz[rx] += siz[ry];
            return true;
        }
        // x を含む根付き木のサイズを求める
        int size(int x) {
            return siz[root(x)];
        }
    };
    
    // union-find木がいくつの連結成分からなるかを返す
    long long partial(UnionFind tree){
        long long n = tree.siz.size();
        vector<bool> seen(n, false);
        long long ans = 0;
        for (long long i = 0; i < n; i++){
            if (seen[tree.root(i)]) continue;
            seen[tree.root(i)] = true;
            ans++;
        }
        return ans;
    }
    
    // 無向グラフGがいくつの連結成分からなるかを返す
    long long partial(Graph &G){
        long long siz = G.size();
        UnionFind ki(siz);
        for (long long i = 0; i < siz; i++){
            long long siz2 = G[i].size();
            for (long long j = 0; j < siz2; j++){
                ki.unite(i, G[i][j]);
            }
        }
        long long ret = partial(ki);
        return ret;
    }
    \end{lstlisting}

    \subsection{BIT  (Fenwick Tree)}
    \begin{lstlisting}
    // 数列a(a[0],a[1],…,a[n-1])についての区間和と点更新を扱う
    // 区間和,点更新,二分探索はO(log{n})
    class BIT {
    public:
        //データの長さ
        ll n;
        //データの格納先
        vector<ll> a;
        //コンストラクタ
        BIT(ll n):n(n),a(n+1,0){}
    
        //a[i]にxを加算する
        void add(ll i,ll x){
            i++;
            if(i==0) return;
            for(ll k=i;k<=n;k+=(k & -k)){
                a[k]+=x;
            }
        }
    
        //a[i]+a[i+1]+…+a[j]を求める
        ll sum(ll i,ll j){
            return sum_sub(j)-sum_sub(i-1);
        }
    
        //a[0]+a[1]+…+a[i]を求める
        ll sum_sub(ll i){
            i++;
            ll s=0;
            if(i==0) return s;
            for(ll k=i;k>0;k-=(k & -k)){
                s+=a[k];
            }
            return s;
        }
    
    //a[0]+a[1]+…+a[i]>=xとなる最小のiを求める(任意のkでa[k]>=0が必要)
        ll lower_bound(ll x){
            if(x<=0){
    //xが0以下の場合は該当するものなし→0を返す
                return 0;
            }else{
                ll i=0;ll r=1;
    // 最大としてありうる区間の長さを取得する
    // n以下の最小の二乗のべき(BITで管理する数列の区間で最大のもの)を求める
                while(r<n) r=r<<1;
    //区間の長さは調べるごとに半分になる
                for(int len=r;len>0;len=len>>1) {
                    //その区間を採用する場合
                    if(i+len<n && a[i+len]<x){
                        x-=a[i+len];
                        i+=len;
                    }
                }
                return i;
            }
        }
    };
    \end{lstlisting}
    \section{Graph}

    \subsection{二部グラフ判定}

    \subsection{深さ優先探索 (再帰関数型)}
    \begin{lstlisting}
    // 深さ優先探索
    vector<bool> seen;
    void dfs(const Graph &G, int v) {
        seen[v] = true; // v を訪問済にする
    
        // v から行ける各頂点 next_v について
        for (auto next_v : G[v]) { 
            // next_v が探索済だったらスルー
            if (seen[next_v]) continue;
            dfs(G, next_v); // 再帰的に探索
        }
    }
    \end{lstlisting}

    \subsection{深さ優先探索 (スタック型)}
    \begin{lstlisting}
    // 深さ優先探索
    stack<int> st;
    st.push(start);
    while (!st.empty()) {
        int v = st.top(); st.pop();
        if (seen[v]) continue;
        seen[v] = true;
        for (auto next_v : G[v]) {
            if (seen[next_v]) continue;
            st.push(next_v);
        }
    }
    \end{lstlisting}

    \subsection{幅優先探索}
    \begin{lstlisting}
    // 幅優先探索
    // 全頂点を「未訪問」に初期化
    vector<int> dist(N, -1); 
    queue<int> que;

    // 初期条件 (頂点 0 を初期ノードとする)
    dist[0] = 0;
    que.push(0); // 0 を橙色頂点にする

    // BFS 開始 (キューが空になるまで探索を行う)
    while (!que.empty()) {
        // キューから先頭頂点を取り出す
        int v = que.front(); 
        que.pop();

        // v から辿れる頂点をすべて調べる
        for (int nv : G[v]) {
            // すでに発見済みの頂点は探索しない
            if (dist[nv] != -1) continue; 

            // 新たな白色頂点 nv について距離情報を更新してキューに追加する
            dist[nv] = dist[v] + 1;
            que.push(nv);
        }
    }
    \end{lstlisting}

    \subsection{ダイクストラ法}
    \begin{lstlisting}
    // 負の重みがない場合の最短経路を求める

    // 辺を表す構造体
    struct Edge{
        long long to;
        long long cost;
        // その他、必要な情報があれば要素を追加
    };
    // 隣接リストを表す型
    using Gpaph=vector<vector<Edge>>;
    // 距離と頂点のペアを表す型
    using Pair = pair<long long, long long>; 
    // 暫定距離を格納する配列
    vector<long long> dist; 
    const long long INF = 1LL << 60;
    
    void dijkstra(const Graph& G, vector<long long>& dist, long long start){
        priority_queue<Pair,vector<Pair>,greater<Pair>> Q;
        dist.assign(G.size(),INF);
        // dist[start]=0をして、qに(0,start)をpush
        Q.emplace(dist[start]=0,start); 
    
        while(!Q.empty()){
        Pair q=Q.top();
        Q.pop();
        long long d=q.first;
        long long v=q.second;
    
        if(d>dist[v]) continue;
    
        for(const auto& edge:G[v]){
            long long nextdist = d+edge.cost;
            if(nextdist<dist[edge.to]){
            Q.emplace(dist[edge.to]=nextdist,edge.to);
            }
        }
        }
    }
    \end{lstlisting}

    \subsection{ベルマンフォード法}
    \begin{lstlisting}
    // 負の重みがある場合の最短経路を求める
    struct Edge {
        long long from;
        long long to;
        long long cost;
    };
    using Edges = vector<Edge>;
    const long long INF = 1LL << 60;
    
    /* bellman_ford(Es,s,t,dis)
        入力: 全ての辺Es, 頂点数V, 開始点 s, 最短経路を記録するdis
        出力: 負の閉路が存在するなら ture
        計算量:O(|E||V|)
        副作用:dis が書き換えられる
    */
    bool bellman_ford(const Edges &Es, int V, int s, vector<long long> &dis) {
        dis.resize(V, INF);
        dis[s] = 0;
        int cnt = 0;
        while (cnt < V) {
            bool end = true;
            for (auto e : Es) {
                if (dis[e.from] != INF && dis[e.from] + e.cost < dis[e.to]) {
                    dis[e.to] = dis[e.from] + e.cost;
                    end = false;
                }
            }
            if (end) break;
            cnt++;
        }
        return (cnt == V);
    }    
    \end{lstlisting}

    \section{探索}
    \subsection{二分探索}
    \begin{lstlisting}
    vector<int> a = { 1,4,4,7,7,8,8,11,13,19};
    // lower_bound:key以上の値が初めて現れる位置
    auto iter = lower_bound(all(a),4);
    // key以上の最小の値を出力
    cout << *iter << endl; // 4
    // key以上の最小の値が初めて現れる位置を出力
    cout << a.begin() - iter << endl; // 1

    // upper_bound:keyより大きい値が初めて現れる位置
    auto iter1 = upper_bound(all(a), 4);
    // keyより大きい最小の値を出力
    cout << *iter1 << endl; // 7
    // keyより大きい最小の値が初出する位置を出力
    cout << a.begin() - iter1 << endl; // 3

    \end{lstlisting}

    \subsection{bit全探索}
    \begin{lstlisting}
    for(int bit = 0; bit < (1 << n); bit++){
        // データ数nのbit全探索
        for(int i = 0; i < n; i++){
            if(bit & (1 << i)){
                // 処理を書く
            }
        }
    }
    \end{lstlisting}

    \subsection{順列全探索}
    \begin{lstlisting}
    vector<int> A(4);
    A = [1, 2, 3, 4];
    do{
        // ここに処理を書く
    }while(next_permutation(A.begin(), A.end()));
    \end{lstlisting}


    \section{その他}
    \subsection{素数判定}
    \begin{lstlisting}
    // エラトステネスの篩 O(N)
    vector<bool> Eratosthenes(int N) {
        // テーブル
        vector<bool> isprime(N+1, true);
    
        // 0, 1 は予めふるい落としておく
        isprime[0] = isprime[1] = false;
    
        // ふるい
        for (int p = 2; p <= N; ++p) {
            // 合成数であるものはスキップする
            if (!isprime[p]) continue;
    
            // p以外のpの倍数から素数ラベルを剥奪
            for (int q = p * 2; q <= N; q += p) {
                isprime[q] = false;
            }
        }
    
        // 1 以上 N 以下の整数が素数かどうか
        return isprime;
    }
    \end{lstlisting}

    \subsection{繰返し二乗法}
    \begin{lstlisting}
    long long pow(long long x, long long n) {
        long long ret = 1;
        while (n > 0) {
            // nの最下位bitが1ならばx^(2^i)をかける
            if (n & 1) ret *= x;  
            x *= x;
            n >>= 1;  // n を1bit 左にずらす
        }
        return ret;
    }
    \end{lstlisting}

    \subsection{余剰を取る繰返し二乗法}
    \begin{lstlisting}
    const int MOD = 1000000007;
    long long pow(long long x, long long n) {
        long long ret = 1;
        while (n > 0) {
            // nの最下位bitが1ならばx^(2^i)をかける
            if (n & 1) ret = ret * x % MOD;  
            x = x * x % MOD;
            n >>= 1;  // n を1bit 左にずらす
        }
        return ret;
    }
    \end{lstlisting}

    \subsection{modの逆元}
    \begin{lstlisting}
    // mod. m での a の逆元 a^{-1} を計算する
    long long modinv(long long a, long long m) {
        long long b = m, u = 1, v = 0;
        while (b) {
            long long t = a / b;
            a -= t * b; swap(a, b);
            u -= t * v; swap(u, v);
        }
        u %= m;
        if (u < 0) u += m;
        return u;
    }
    \end{lstlisting}

    \subsection{modintの使い方}
    \begin{lstlisting}
    using mint = modint998244353;

    int main(){
        mint ans = 4321;
        ans /= 9876;
        // 4321 / 9876 mod 998244353を出力
        cout << ans.val() << endl;
    }
    \end{lstlisting}


    \newpage

    \part{Python}
    % Pythonのセクションに切り替える
    \lstset{style=python}
    \setcounter{section}{0}
    \subsection{Python Execution}
    \begin{lstlisting}
    python3 main.py < input.txt > output.txt
    \end{lstlisting}

    \section{データ構造}
    \subsection{UnionFind}
    \begin{lstlisting}
    class UnionFind:
    def __init__(self, n):
        self.par = [-1] * n
        self.rank = [0] * n
        self.siz = [1] * n

    def root(self, x):
        if self.par[x] == -1:
            return x
        else:
            self.par[x] = self.root(self.par[x])
            return self.par[x]

    def issame(self, x, y):
        return self.root(x) == self.root(y)

    def unite(self, x, y):
        rx = self.root(x)
        ry = self.root(y)
        if rx == ry:
            return False
        if self.rank[rx] < self.rank[ry]:
            rx, ry = ry, rx
        self.par[ry] = rx
        if self.rank[rx] == self.rank[ry]:
            self.rank[rx] += 1
        self.siz[rx] += self.siz[ry]
        return True

    def size(self, x):
        return self.siz[self.root(x)]


    def partial(tree):
        n = len(tree.siz)
        seen = [False] * n
        ans = 0
        for i in range(n):
            if not seen[tree.root(i)]:
                seen[tree.root(i)] = True
                ans += 1
        return ans


    def partial_graph(G):
        siz = len(G)
        uf = UnionFind(siz)
        for i in range(siz):
            for j in G[i]:
                uf.unite(i, j)
        return partial(uf)


    # 使用例
    # グラフGを隣接リストとして表現
    G = [
        [1, 2],  # ノード0の隣接ノード
        [0, 2],  # ノード1の隣接ノード
        [0, 1],  # ノード2の隣接ノード
        [4],     # ノード3の隣接ノード
        [3]      # ノード4の隣接ノード
    ]

    # グラフの連結成分の数を計算
    num_connected_components = partial_graph(G)
    print(num_connected_components)  # 出力: 2
        
    \end{lstlisting}

    \section{Graph}
    \subsection{深さ優先探索(再帰関数型)}
    \begin{lstlisting}
    def dfs(G, v, seen):
    seen[v] = True  # v を訪問済にする
    
    # v から行ける各頂点 next_v について
    for next_v in G[v]:
        # next_v が探索済だったらスルー
        if seen[next_v]:
            continue
        dfs(G, next_v, seen)  # 再帰的に探索
    
    
    # 使用例
    # グラフGを隣接リストとして表現
    G = [
        [1, 2],  # ノード0の隣接ノード
        [0, 2],  # ノード1の隣接ノード
        [0, 1, 3],  # ノード2の隣接ノード
        [2, 4],  # ノード3の隣接ノード
        [3]      # ノード4の隣接ノード
    ]
    
    # 頂点の訪問状態を保持するリスト
    seen = [False] * len(G)
    
    # ノード0からDFSを開始
    dfs(G, 0, seen)
    
    # 結果の出力
    print(seen)  
    # 出力: [True, True, True, True, True]
    \end{lstlisting}

    \subsection{深さ優先探索(スタック型)}
    \begin{lstlisting}
    def dfs_iterative(G, start):
    # 頂点の訪問状態を保持するリスト
    seen = [False] * len(G)
    # スタックの初期化
    st = [start]
    
    while st:
        v = st.pop()
        if seen[v]:
            continue
        seen[v] = True
        for next_v in G[v]:
            if not seen[next_v]:
                st.append(next_v)
    
    return seen
    
    # 使用例
    # グラフGを隣接リストとして表現
    G = [
        [1, 2],  # ノード0の隣接ノード
        [0, 2],  # ノード1の隣接ノード
        [0, 1, 3],  # ノード2の隣接ノード
        [2, 4],  # ノード3の隣接ノード
        [3]      # ノード4の隣接ノード
    ]
    
    # ノード0からDFSを開始
    seen = dfs_iterative(G, 0)
    
    # 結果の出力
    print(seen)  # 出力: [True, True, True, True, True]
    
    \end{lstlisting}

    \subsection{幅優先探索}
    \begin{lstlisting}
    from collections import deque

    def bfs(G, start):
        N = len(G)
        # 全頂点を「未訪問」に初期化
        dist = [-1] * N
        que = deque()
    
        # 初期条件 (頂点 start を初期ノードとする)
        dist[start] = 0
        que.append(start)  # start を橙色頂点にする
    
        # BFS 開始 (キューが空になるまで探索を行う)
        while que:
            # キューから先頭頂点を取り出す
            v = que.popleft()
    
            # v から辿れる頂点をすべて調べる
            for nv in G[v]:
                # すでに発見済みの頂点は探索しない
                if dist[nv] != -1:
                    continue
    
                # 新たな白色頂点 nv について距離情報を更新してキューに追加する
                dist[nv] = dist[v] + 1
                que.append(nv)
    
        return dist
        
        # 使用例
        # グラフGを隣接リストとして表現
        G = [
            [1, 2],  # ノード0の隣接ノード
            [0, 2],  # ノード1の隣接ノード
            [0, 1, 3],  # ノード2の隣接ノード
            [2, 4],  # ノード3の隣接ノード
            [3]      # ノード4の隣接ノード
        ]
        
        # ノード0からBFSを開始
        distances = bfs(G, 0)
        
        # 結果の出力
        print(distances)  # 出力: [0, 1, 1, 2, 3]
        
    \end{lstlisting}

    \subsection{ダイクストラ法}
    \begin{lstlisting}
    import heapq

    class Edge:
        def __init__(self, to, cost):
            self.to = to
            self.cost = cost
    
    def dijkstra(G, start):
        INF = float('inf')
        dist = [INF] * len(G)
        dist[start] = 0
        priority_queue = []
        heapq.heappush(priority_queue, (0, start))
    
        while priority_queue:
            d, v = heapq.heappop(priority_queue)
    
            if d > dist[v]:
                continue
    
            for edge in G[v]:
                nextdist = d + edge.cost
                if nextdist < dist[edge.to]:
                    dist[edge.to] = nextdist
                    heapq.heappush(priority_queue, (nextdist, edge.to))
    
        return dist
        
        # 使用例
        # グラフGを隣接リストとして表現
        G = [
            [Edge(1, 2), Edge(2, 4)],  # ノード0の隣接ノードとコスト
            [Edge(2, 1), Edge(3, 7)],  # ノード1の隣接ノードとコスト
            [Edge(3, 3)],              # ノード2の隣接ノードとコスト
            []                         # ノード3の隣接ノードとコスト
        ]
        
        # ノード0からダイクストラ法を開始
        start_node = 0
        distances = dijkstra(G, start_node)
        
        # 結果の出力
        print(distances)  # 出力: [0, 2, 3, 6]
        
    \end{lstlisting}

    \subsection{ベルマンフォード法}
    \begin{lstlisting}
    class Edge:
    def __init__(self, from_node, to_node, cost):
        self.from_node = from_node
        self.to_node = to_node
        self.cost = cost

def bellman_ford(edges, V, start):
    INF = float('inf')
    dist = [INF] * V
    dist[start] = 0

    for i in range(V):
        updated = False
        for edge in edges:
            if dist[edge.from_node] != INF and dist[edge.from_node] + edge.cost < dist[edge.to_node]:
                dist[edge.to_node] = dist[edge.from_node] + edge.cost
                updated = True
        if not updated:
            break

    # Check for negative weight cycles
    for edge in edges:
        if dist[edge.from_node] != INF and dist[edge.from_node] + edge.cost < dist[edge.to_node]:
            return True, dist  # Negative weight cycle detected

    return False, dist
    
    # 使用例
    edges = [
        Edge(0, 1, 2),
        Edge(0, 2, 4),
        Edge(1, 2, 1),
        Edge(1, 3, 7),
        Edge(2, 3, 3),
        Edge(3, 4, -5),
        Edge(4, 1, 2)
    ]
    
    V = 5  # グラフの頂点数
    start_node = 0
    has_negative_cycle, distances = bellman_ford(edges, V, start_node)
    
    if has_negative_cycle:
        print("負の閉路があります")
    else:
        print("最短距離:", distances)
    
    \end{lstlisting}

    \section{探索}

    \subsection{二分探索}
    
    \begin{lstlisting}
    import bisect

    a = [1, 4, 4, 7, 7, 8, 8, 11, 13, 19]
    
    # lower_bound: key以上の値が初めて現れる位置
    key = 4
    iter_idx = bisect.bisect_left(a, key)
    print(a[iter_idx])  # 4
    print(iter_idx)     # 1
    
    # upper_bound: keyより大きい値が初めて現れる位置
    iter1_idx = bisect.bisect_right(a, key)
    print(a[iter1_idx])  # 7
    print(iter1_idx)     # 3
    \end{lstlisting}

    \subsection{bit全探索}
    \begin{lstlisting}
    n = 3  # データ数

    for bit in range(1 << n):
        # データ数nのbit全探索
        for i in range(n):
            if bit & (1 << i):
                # 処理を書く
                print(f"bit: {bin(bit)}, i: {i}")
    \end{lstlisting}

    \subsection{順列全探索}
    \begin{lstlisting}
    import itertools

    A = [1, 2, 3, 4]
    
    # permutationsを使って全ての順列を生成
    for perm in itertools.permutations(A):
        # ここに処理を書く
        print(perm)
    \end{lstlisting}

    \section{その他}
    \subsection{素数判定}
    \begin{lstlisting}
    def eratosthenes(N):
    # テーブル
    isprime = [True] * (N + 1)
    
    # 0, 1 は予めふるい落としておく
    isprime[0] = isprime[1] = False
    
    # ふるい
    for p in range(2, N + 1):
        # 合成数であるものはスキップする
        if not isprime[p]:
            continue
        
        # p以外のpの倍数から素数ラベルを剥奪
        for q in range(p * 2, N + 1, p):
            isprime[q] = False
    
    # 1 以上 N 以下の整数が素数かどうか
    return isprime
    
    # 使用例
    N = 30
    isprime = eratosthenes(N)
    print(isprime)
    
    # 素数のリストを取得
    primes = [i for i, prime in enumerate(isprime) if prime]
    print(primes)
    \end{lstlisting}


\end{multicols*}

\end{document}
